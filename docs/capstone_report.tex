\documentclass[12pt]{article}

% ===========================
% Packages
% ===========================
\usepackage[margin=1in]{geometry}
\usepackage{hyperref}
\usepackage{graphicx}
\usepackage{array}
\usepackage{longtable}
\usepackage{enumitem}
\usepackage{titlesec}
\usepackage{fancyhdr}
\usepackage{amsmath}

% ===========================
% Formatting
% ===========================
\titleformat{\section}{\large\bfseries}{\thesection}{0.75em}{}
\titleformat{\subsection}{\normalsize\bfseries}{\thesubsection}{0.75em}{}
\titleformat{\subsubsection}{\normalsize\itshape}{\thesubsubsection}{0.75em}{}

\pagestyle{fancy}
\fancyhf{}
\rhead{StepHabit Capstone}
\lhead{Comprehensive Report}
\cfoot{\thepage}

\begin{document}

% ===========================
% Title Page
% ===========================
\begin{titlepage}
  \centering
  {\Huge StepHabit Capstone Report\\[1em]}
  {\Large A comprehensive technical and product deep dive}\\[2em]
  {\large Prepared by: Engineering Team}\\[1em]
  {\large Date: \today}\\[3em]
  \vfill
  {\large Abstract}\\[0.5em]
  StepHabit is a full-stack habit-building and productivity platform designed to help
  users convert long-term goals into consistent daily action. The system integrates
  habit tracking, task management, intelligent scheduling, and AI-assisted coaching
  within a unified experience. This report presents a detailed technical and product-level
  analysis of StepHabit, covering system architecture, data modeling, backend services,
  frontend workflows, and implementation decisions. The document serves both as a formal
  capstone submission and as a long-term reference for future contributors and system
  extensions.
  \vfill
\end{titlepage}

\newpage
\tableofcontents
\newpage

% ===========================
% Chapter 1
% ===========================
\section{Introduction}
\subsection{System Overview}
StepHabit is a secure, AI-assisted productivity and habit-building platform designed
to help users transform long-term goals into consistent daily routines. Built with a
strong emphasis on structure, motivation, and sustainability, the system enables
users to create, manage, and track habits, tasks, and schedules within a unified
environment. A secure registration and authentication process ensures account
integrity, while personalized dashboards allow users to monitor progress, streaks,
and achievements over time.

Once authenticated, users can define habits with daily goals, manage tasks with
durations and deadlines, and organize their time through calendar-based planning
and busy-block scheduling. The platform provides visual feedback through planners,
charts, and progress logs, helping users clearly understand how their daily actions
align with broader objectives. Notifications, reminders, and achievement milestones
reinforce engagement and encourage consistency.

Beyond traditional productivity tools, StepHabit integrates artificial intelligence to
provide personalized coaching and guidance. An AI-driven assistant analyzes user
input and context to generate structured habit plans, refine habit ideas, and offer
actionable suggestions. By maintaining assistant memory tied to individual users,
the system delivers increasingly relevant feedback and supports reflective,
goal-oriented decision making over time.

User data is stored securely and is accessible only to authenticated users, with all
core features scoped to individual accounts to ensure privacy. Social features such
as friendships, messaging, and group challenges are implemented within controlled
boundaries, allowing users to engage in accountability-driven interactions without
compromising personal data. Calendar integrations further enhance the experience
by preventing scheduling conflicts between habits, tasks, and external commitments.

In summary, StepHabit combines secure account management, structured planning,
social accountability, and AI-powered coaching to deliver a comprehensive and
intelligent productivity experience. The platform demonstrates how modern web
technologies and artificial intelligence can be thoughtfully integrated to support
sustainable habit formation and long-term personal growth.

\subsection{Objectives}
The primary objective of this project is to design and develop a secure, AI-assisted
productivity and habit-building platform that enables users to translate long-term
goals into consistent daily actions. The system aims to provide a structured yet
flexible environment in which users can create, manage, and track habits, tasks, and
schedules, while receiving intelligent guidance to improve planning and execution.
Through AI-powered coaching, the platform supports users in refining habit ideas,
breaking goals into actionable steps, and maintaining motivation over time.

In addition, the project seeks to deliver an intuitive and user-friendly interface,
secure account management through authenticated access, and personalized dashboards
that visualize progress, streaks, and achievements. By combining scheduling tools,
progress tracking, and social accountability mechanisms, StepHabit encourages
sustainable behavior change rather than short-term productivity bursts.

Secondary objectives include building a scalable full-stack web application using
modern frameworks, ensuring data security and user privacy, and establishing a
robust architectural foundation that can support future extensions such as mobile
clients, advanced analytics, or enhanced AI-driven personalization.
\subsection{Scope}
The primary objective of this project is to design and develop a secure, AI-assisted
productivity and habit-building platform that enables users to translate long-term
goals into consistent daily actions. The system aims to provide a structured yet
flexible environment in which users can create, manage, and track habits, tasks, and
schedules, while receiving intelligent guidance to improve planning and execution.
Through AI-powered coaching, the platform supports users in refining habit ideas,
breaking goals into actionable steps, and maintaining motivation over time.

In addition, the project seeks to deliver an intuitive and user-friendly interface,
secure account management through authenticated access, and personalized dashboards
that visualize progress, streaks, and achievements. By combining scheduling tools,
progress tracking, and social accountability mechanisms, StepHabit encourages
sustainable behavior change rather than short-term productivity bursts.

Secondary objectives include building a scalable full-stack web application using
modern frameworks, ensuring data security and user privacy, and establishing a
robust architectural foundation that can support future extensions such as mobile
clients, advanced analytics, or enhanced AI-driven personalization.
\subsection{Scope of the Project}
\subsection{Structure of the Report}

\newpage

% ===========================
% Chapter 2
% ===========================
\section{System Architecture}
\subsection{Overview of System Components}
The StepHabit platform is designed using a modular system architecture, in which each
major component is responsible for a clearly defined set of responsibilities. This
architectural approach improves clarity, maintainability, and scalability, while
allowing individual components to evolve independently. Together, these components
form a cohesive, secure, and intelligent productivity system that supports habit
formation, task management, and goal-oriented behavior.

The frontend application serves as the primary interaction layer between users and the
system. It provides interfaces for user registration and authentication, habit and task
creation, calendar-based planning, and progress tracking. Through dashboards,
planners, and visual analytics, users can monitor habits, streaks, achievements, and
upcoming tasks. Social features such as messaging, group challenges, and notifications
are also accessible through the frontend, enabling accountability-driven engagement in
a user-friendly and responsive interface.

The backend server implements the core business logic and coordinates communication
between the frontend, the database, and external services. It manages authentication
and authorization, enforces data ownership and privacy rules, and processes all user
actions, including habit updates, task scheduling, progress logging, notifications, and
community interactions. The backend also exposes RESTful APIs that provide a stable
and extensible interface for current and future clients.

The database layer is responsible for persistent data storage and integrity. It stores
user profiles, habits, tasks, schedules, progress records, achievements, notifications,
assistant memory, and social relationships. The relational schema is designed to support
clear ownership boundaries between users and their data, while enabling efficient
queries for analytics, dashboards, and planner views.

An AI-powered coaching component enhances the platform by providing intelligent
guidance and personalization. This service analyzes user-provided habit ideas and
contextual data to generate structured habit plans, refine goal descriptions, and offer
actionable suggestions. Assistant memory is maintained on a per-user basis, allowing
the system to deliver increasingly relevant and consistent coaching over time. AI
outputs are evaluated and stored where appropriate to support reflection and future
interactions.

Together, these components form a robust and extensible system that integrates
structured planning, progress visibility, social accountability, and AI-driven guidance.
StepHabit combines modern web technologies with intelligent feedback mechanisms to
deliver a productivity platform that supports sustainable habit formation and long-term
personal growth.
\subsection{Technology Stack and Frameworks}
The StepHabit platform utilizes a modern, modular technology stack designed to ensure
performance, scalability, maintainability, and ease of future extension across all
layers of the system. Each technology was selected to support rapid development while
maintaining architectural clarity and long-term sustainability.

Frontend
React (Vite):
A modern JavaScript library used to build responsive and interactive user interfaces.
Vite is employed as the build tool to enable fast development cycles and optimized
production bundles.
CoreUI:
A component-based UI framework used to implement consistent layouts, navigation
structures, dashboards, and data visualization elements across the application.

Backend
Node.js with Express:
Node.js serves as the runtime environment for the backend, while Express provides a
lightweight and flexible framework for handling HTTP requests, routing, and middleware
composition.
Sequelize ORM:
Used to manage relational data interactions with PostgreSQL, providing model
definitions, associations, and schema synchronization.
JWT (JSON Web Tokens):
Employed for secure authentication and authorization, ensuring that protected API
endpoints are accessible only to authenticated users.
Bcrypt:
Used for secure password hashing and credential protection.
Socket-Based Messaging (Planned):
The architecture supports real-time features such as notifications and messaging,
allowing future integration of WebSocket-based communication if required.

Database
PostgreSQL:
A robust open-source relational database used to store structured data, including user
profiles, habits, tasks, schedules, progress logs, achievements, notifications, and
social relationships. PostgreSQL’s reliability and relational capabilities support
data integrity and complex querying needs.

AI Integration
LangChain with Large Language Models (Anthropic):
Used to implement AI-assisted habit coaching, including habit plan generation, idea
refinement, and contextual guidance. LangChain provides an abstraction layer that
simplifies prompt management, model configuration, and future AI provider replacement.

Environment-Based Configuration:
Sensitive credentials and environment-specific variables are managed through
environment files, supporting secure and flexible configuration across development
and production environments.
\subsection{Development and DevOps Tools}
The development of the StepHabit platform relied on a set of modern development and
DevOps tools to support efficient implementation, testing, collaboration, and
deployment. These tools contributed to code quality, system reliability, and
developer productivity throughout the project lifecycle.

Visual Studio Code (VS Code):
Used as the primary integrated development environment for both frontend and backend
development. VS Code provides rich language support, debugging tools, and extensions
that streamline full-stack development.

Postman:
Employed extensively during backend development for testing RESTful API endpoints,
validating request and response payloads, and debugging authentication and business
logic workflows.

Git and GitHub:
Used for version control, source code management, and collaboration. Git enables
systematic tracking of code changes, while GitHub supports repository hosting, issue
tracking, and pull-request-based development workflows.

Docker and Docker Desktop:
Used to containerize the application’s backend services, frontend client, and
database, ensuring consistent environments across local development and deployment.
Docker Compose simplifies orchestration of multi-container setups.

Navicat:
Utilized as a database administration and visualization tool for managing the
PostgreSQL database, inspecting tables and relationships, and validating data
integrity during development and testing.

ESLint and Prettier:
Applied to enforce consistent coding standards, detect potential issues early, and
maintain readability and maintainable code across the project.

Email and Notification Monitoring Tools:
Used to test and monitor email-based features such as account verification and
notification delivery, ensuring reliability and correctness of user-facing
communication workflows.

API Documentation Tools (Swagger UI):
Used to document and interactively test API endpoints, improving transparency of the
API surface and simplifying development, debugging, and future system integration.


\newpage

% ===========================
% Chapter 3
% ===========================
\section{Database Design}
\subsection{Database Choice}
PostgreSQL was chosen for this project because of its excellent support for relational data,
adherence to ACID principles, and support for TypeORM integration. PostgreSQL's support for
complex querying and indexing provides an effective solution to managing the complex
relationships between users, posts, and interactions (likes, comments).
\subsection{Requirements Gathering Process}
\subsection{Functional Requirements}
\subsection{Non-Functional Requirements}
\subsection{Success Criteria}

\newpage

% ===========================
% Chapter 4
% ===========================
\section{System Architecture}
\subsection{High-Level Architecture}
\subsection{Technology Stack}
\subsection{Backend Architecture}
\subsection{Frontend Architecture}
\subsection{AI and Intelligent Components}
\subsection{Deployment Architecture}

\newpage

% ===========================
% Chapter 5
% ===========================
\section{Backend Design and Implementation}
\subsection{Application Entry Point}
\subsection{Routing and Middleware}
\subsection{Controller Layer}
\subsection{Service Layer}
\subsection{Database Access Layer}
\subsection{Error Handling Strategy}

\newpage

% ===========================
% Chapter 6
% ===========================
\section{Data Modeling}
\subsection{Database Design Overview}
\subsection{Core Entities}
\subsection{Relationships and Associations}
\subsection{Data Integrity Constraints}
\subsection{Data Lifecycle Management}

\newpage

% ===========================
% Chapter 7
% ===========================
\section{API Design}
\subsection{REST API Principles}
\subsection{Authentication and Authorization APIs}
\subsection{Habit and Task APIs}
\subsection{Scheduling APIs}
\subsection{AI Service APIs}
\subsection{Messaging and Notification APIs}

\newpage

% ===========================
% Chapter 8
% ===========================
\section{AI and Coaching System}
\subsection{AI System Overview}
\subsection{Prompt Engineering Strategy}
\subsection{Assistant Memory Design}
\subsection{Safety and Guardrails}
\subsection{Future AI Extensions}

\newpage

% ===========================
% Chapter 9
% ===========================
\section{Frontend Design and Implementation}
\subsection{UI Framework and Design System}
\subsection{Application Layout and Navigation}
\subsection{Routing and State Management}
\subsection{Core Screens and User Flows}
\subsection{Accessibility and Responsiveness}

\newpage

% ===========================
% Chapter 10
% ===========================
\section{Security and Privacy}
\subsection{Authentication Mechanisms}
\subsection{Authorization Model}
\subsection{Input Validation}
\subsection{Data Privacy and Compliance}
\subsection{Secrets Management}

\newpage

% ===========================
% Chapter 11
% ===========================
\section{Scheduling and Productivity Workflows}
\subsection{Habits vs Tasks}
\subsection{Smart Scheduling}
\subsection{Calendar Integration}
\subsection{Progress Tracking}
\subsection{Achievements and Motivation}

\newpage

% ===========================
% Chapter 12
% ===========================
\section{Community and Communication}
\subsection{Friendship Model}
\subsection{Group Challenges}
\subsection{Messaging System}
\subsection{Notifications}
\subsection{Moderation Policies}

\newpage

% ===========================
% Chapter 13
% ===========================
\section{Testing and Quality Assurance}
\subsection{Testing Strategy}
\subsection{Unit Testing}
\subsection{Integration Testing}
\subsection{End-to-End Testing}
\subsection{Performance Testing}

\newpage

% ===========================
% Chapter 14
% ===========================
\section{Deployment and Operations}
\subsection{Environment Configuration}
\subsection{Containerization}
\subsection{CI/CD Pipeline}
\subsection{Scaling Considerations}
\subsection{Monitoring and Logging}

\newpage

% ===========================
% Chapter 15
% ===========================
\section{Evaluation and Metrics}
\subsection{Product Metrics}
\subsection{Technical Metrics}
\subsection{AI Performance Metrics}
\subsection{User Engagement Analysis}

\newpage

% ===========================
% Chapter 16
% ===========================
\section{Future Work}
\subsection{Feature Enhancements}
\subsection{AI Improvements}
\subsection{Scalability Roadmap}
\subsection{Research Directions}

\newpage

% ===========================
% Chapter 17
% ===========================
\section{Ethical Considerations}
\subsection{Responsible AI Use}
\subsection{User Privacy}
\subsection{Bias and Fairness}
\subsection{Transparency and Consent}

\newpage

% ===========================
% Chapter 18
% ===========================
\section{Conclusion}
\subsection{Summary of Contributions}
\subsection{Lessons Learned}
\subsection{Final Remarks}

\end{document}